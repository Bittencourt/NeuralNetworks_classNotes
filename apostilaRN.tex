%%This is a very basic article template.
%%There is just one section and two subsections.
\documentclass[a4paper,10pt]{article}

\usepackage[brazilian]{babel}
\usepackage[latin1]{inputenc}
\usepackage[T1]{fontenc}
%\usepackage[pdftex]{graphicx}
%\usepackage{tikz}
%\usetikzlibrary{shapes,arrows}
\let\stdsection\section \renewcommand\section{\newpage\stdsection}  

\usepackage{pst-all}
\usepackage{multido}
\usepackage{tikz}
\usetikzlibrary{scopes,matrix,positioning}
\tikzset{
  mymx/.style={matrix of math nodes,nodes=myball,column sep=4em,row sep=-1ex},
  myball/.style={draw,circle,inner sep=0pt},
  mylabel/.style={midway,sloped,fill=white,inner sep=1pt,outer sep=1pt,below,
    execute at begin node={$\scriptstyle},execute at end node={$}},
  plain/.style={draw=none,fill=none},
  sel/.append style={fill=green!10},
  prevsel/.append style={fill=red!10},
  route/.style={-latex,thick},
  selroute/.style={route,blue!50!green}
}

\newcommand{\HRule}{\rule{\linewidth}{0.5mm}}

\begin{document}

\begin{titlepage}

\begin{center}


% Upper part of the page
%\includegraphics[width=0.15\textwidth]{./logo}\\[1cm]    

\textsc{\LARGE PEE - COPPE}\\[1.5cm]

\textsc{\Large Notas de Aula}\\[0.5cm]


% Title
\HRule \\[0.4cm]
{ \huge \bfseries CPE722 - Redes Neurais n�o Supervisionadas e
Clusteriza��o\\ 2012.3}\\[0.4cm]

\HRule \\[1.5cm]

% Author and supervisor
\begin{minipage}{0.4\textwidth}
\begin{flushleft} \large
\emph{Autor:}\\
Pedro Bittencourt
\end{flushleft}
\end{minipage}
\begin{minipage}{0.4\textwidth}
\begin{flushright} \large
\emph{Professor:} \\
L.P. Cal�ba
\end{flushright}
\end{minipage}

\vfill

% Bottom of the page
{\large \today}

\end{center}

\end{titlepage}

\section{Programa}

\subsection{Parte I - Complemento de Redes Feedfoward supervisionadas}

\begin{itemize}
  \item Poda 
  \item Outras redes feedfoward 
  \item RNs e s�ries temporais
  \item RNs em PCAs generalizadas
\end{itemize}

\subsection{Parte IIa - Agrupamentos(Clusteriza��o)}


\begin{itemize}
  \item Introdu��o e Motiva��o ao Uso
  \item Classificadores e Classifica��o por Similaridade
  \item Aprendizado Supervisionado e N�o Supervisionado
  \item Clusteriza��o e propriedades das classes
  \item Pr�-processamento
  \item Gera��o de padr�es e classifica��o
  \item Clusteriza��o, processos cl�ssicos
\end{itemize}


\subsection{Parte IIb - Redes Neurais com treinamento n�o supervisionado}

\begin{itemize}
  \item Redes Neurais em Clusteriza��o
  \item Quantiza��o Vetorial
  \item Camada de Kohonen Simplificada
  \item Rede ART
  \item Counterpropagation
  \item Classes n�o esf�ricas
  \item SOM
  \item Demonstrativos e exemplos
\end{itemize}

\section{Refer�ncias bibliogr�ficas do curso}

\subsection{Livros}

\begin{itemize}
 \item  Wasserman, P. - 'Neural Computing', Van Nostrand Reinhold, 1989, Cap 4,
 8.
\item Ivan Silva, I.; Spatti, D. e Flauzini, R. - 'Redes Neurais Artificiais para 
Engenharia e Ci�ncias Aplicadas', Artliber, 2010, cap 8-10.
\item Duda, R.O., Hart, P.E., 'Pattern Classification and Scene Anal.', Wiley,
1973,Cap 6.
\item Haykin, S., 'Neural Networks and Learning machines', Pearson, 2009 ou 'Redes 
Neurais, Teoria e Pr�tica', Bookman, 2001, Cap. 8.1, 8.2, Cap. 9.  
\item Duda, R.O.; Hart, P.E.; Stork, D.G., 'Pattern Classification', Wiley, 2001, Cap 10. 
\item Gersho, A., Gray, R.M. - 'Vector Quantization and Signal Compression',
Kluwer, 1992, Part III.
\end{itemize}

\subsection{Software}
\begin{itemize}
  \item Neuralworks, Neuroshell
  \item Toolboxes: Matlab, Statistica
  \item SNNS (Stuttgart Neural Network Simulator)
  \item WEKA (Waikato Environment for Knowledge Analysis)
\end{itemize}

\section{Aula 1 - Revis�o de Redes Neurais Supervisionadas}
Esta sess�o cobre algumas estruturas de Redes Neurais com treinamento
supervisionado e algoritmos estruturais.

\subsection{Algoritmo Destrutivo - Poda ou \textit{Prunning}}
O algoritmo de poda reduz a estrutura da rede reduzindo suas sinapses e/ou
neur�nios. Este algoritmo apresenda certas vantagens:
\begin{itemize}
  \item Para aplica��es em tempo real uma rede mais simpls � vantanjosa para
  reduzir a quantidade de opera��es e velocidade de execu��o.
  \item Com a poda � poss�vel extrair a l�gica da rede
\end{itemize}
\begin{tikzpicture}
  \matrix[mymx] (mx) {
    |[plain]| x_1 &|[prevsel,yshift=4ex]| f(e) & f_4(e) \\
    \:&|[prevsel]| f(e) && y &|[plain]| y \\
    |[plain]| x_n &|[sel,yshift=-4ex]| f(e) & f_5(e) \\
  };
  {[route]
    \foreach \y in {1,3} {
      \draw (mx-\y-1) -- (mx-1-2);
      \draw (mx-\y-1) -- (mx-2-2);
      \draw (mx-\y-3) -- (mx-2-4); }
    \foreach \y in {1,2,3} {
      \draw (mx-\y-2) -- (mx-1-3);
      \draw (mx-\y-2) -- (mx-3-3); }
    \draw (mx-2-4) -- (mx-2-5);
  }
  {[selroute]
    \draw (mx-1-1) -- (mx-3-2) node[mylabel] { W_{1,3} };
    \draw (mx-3-1) -- (mx-3-2) node[mylabel] { W_{n,3} };
  }
  \node[above right=of mx.center] {$ y_3 = f_3(w_{(x1)3}x_1 + w_{(x2)3} x_2) $};
\end{tikzpicture}
\subsection{Algoritmos Construtivos}
\subsubsection{SONN: Self Organizes Neural Network}
\subsubsection{Classificador Tilling}
\subsubsection{Classificador Hier�rquico}
\subsubsection{Rede de Perceptrons}
\section{Aula 2 - PCA generalizado com Redes Neurais}

\begin{tikzpicture}
  \matrix[mymx] (mx) {
    &|[prevsel]| f_1(e) \\
    |[plain]| x_1 && f_4(e) \\
    &|[sel]| f_2(e) && f_6(e) & |[plain]| y \\
    |[plain]| x_2 && f_5(e) \\
    & f_3(e) \\
  };
  {[route]
    \foreach \y in {2,4} {
      \draw (mx-\y-1) -- (mx-1-2);
      \draw (mx-\y-1) -- (mx-5-2);
      \draw (mx-\y-3) -- (mx-3-4); }
    \foreach \y in {1,3,5} {
      \draw (mx-\y-2) -- (mx-2-3);
      \draw (mx-\y-2) -- (mx-4-3); }
    \draw (mx-3-4) -- (mx-3-5);
  }
  {[selroute]
    \draw (mx-2-1) -- (mx-3-2) node[mylabel,above] { W_{(x1)2} };
    \draw (mx-4-1) -- (mx-3-2) node[mylabel] { W_{(x2)2} };
  }
  \node[above right=of mx.center]  {$ y_2 = f_2 (w_{(x1)2} x_1 + w_{(x2)2} x_2) $};
\end{tikzpicture}

\begin{tikzpicture}
  \matrix[mymx] (mx) {
    |[plain]| x_1 &|[prevsel,yshift=4ex]| f_1(e) & f_4(e) \\
    &|[prevsel]| f_2(e) && f_6(e) &|[plain]| y \\
    |[plain]| x_2 &|[sel,yshift=-4ex]| f_3(e) & f_5(e) \\
  };
  {[route]
    \foreach \y in {1,3} {
      \draw (mx-\y-1) -- (mx-1-2);
      \draw (mx-\y-1) -- (mx-2-2);
      \draw (mx-\y-3) -- (mx-2-4); }
    \foreach \y in {1,2,3} {
      \draw (mx-\y-2) -- (mx-1-3);
      \draw (mx-\y-2) -- (mx-3-3); }
    \draw (mx-2-4) -- (mx-2-5);
  }
  {[selroute]
    \draw (mx-1-1) -- (mx-3-2) node[mylabel] { W_{(x1)3} };
    \draw (mx-3-1) -- (mx-3-2) node[mylabel] { W_{(x2)3} };
  }
  \node[above right=of mx.center] {$ y_3 = f_3(w_{(x1)3}x_1 + w_{(x2)3} x_2) $};
\end{tikzpicture}

\psset{arrowscale=4}
\def\pscolhookiii{\global\pscolsep=0.5cm}
\def\pscolhookiv{\global\pscolsep=2cm}
\begin{psmatrix}[colsep=2,rowsep=2,mnode=circle]
  [name=N11] N11 & [name=N12] N12 & [name=N13] N13 & [name=N14] N14 \\[0pt]
  [name=N21] N21 & [name=N22] N22 & [name=N23] N23 & [name=N24] N24 \\[2cm]
  [name=N31] N31 & [name=N32] N32 & [name=N33] N33 & [name=N34] N34
\end{psmatrix}
% Node connections
\ncline[ArrowInside=->,ArrowInsideNo=5,%
	linecolor=red]{N11}{N33}
\ncline[ArrowInside=-|,ArrowInsidePos=0.75,%
	linecolor=blue]{<->}{N23}{N34}
\nccurve[angleB=90,ArrowInside={-]},ArrowInsidePos=0.6666,%
	linecolor=cyan]{N13}{N14}
\nccurve[angle=-90,ArrowInside=->,ArrowInsidePos=0.5,%
	linecolor=green]{N31}{N33}
\nccurve[angleA=90,angleB=180,ArrowInside=->,ArrowInsidePos=0.4,%
	linecolor=gray]{N31}{N24}
\end{document}
